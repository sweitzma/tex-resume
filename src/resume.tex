\documentclass{article}[14pt]
\usepackage[utf8]{inputenc}
\usepackage{tikz,lipsum,lmodern}
\usepackage[most]{tcolorbox}
\usepackage{enumitem}
\usepackage[margin=2cm]{geometry}
\usepackage{fontawesome} % ICONS: http://ctan.math.illinois.edu/fonts/fontawesome/doc/fontawesome.pdf
\usepackage{amssymb}
\usepackage[hidelinks]{hyperref}

\setlength\parindent{0pt}

\NewDocumentCommand{\colorrule}{O{.4pt}m}{%
}

\newcommand{\skill}[1]{
  \hfill
  \foreach \n in {1,...,#1}{{\normalsize$\bullet$\hspace{1pt}}}
  \vspace{4pt}
  \newline
}

% color pallet
\definecolor{C1}{HTML}{0B9F83}
\definecolor{C2}{HTML}{42858B}
\definecolor{C3}{HTML}{786C93}
\definecolor{C4}{HTML}{AF529B}
\definecolor{C4.5}{HTML}{CB469F}
\definecolor{C5}{HTML}{E639A3}

%%%%%
% note that the opacity creates the two colors
% in the tcolorbox
%%%%%

% $1:= Title
% $2:= Color
\newenvironment{cvsection}[2] {
  \begin{tcolorbox}[
    standard jigsaw, % make background transparent
    opacityback=0,   % make background transparent
    top=0mm,         % size of top boarder
    bottom=0mm,      % size of bottom boarder
    boxrule=0pt,     % boarder of sort
    enhanced,        % enable frame code
    frame code={
      \path[draw=#2,line width=3pt]
        (frame.south west) --
        ([yshift=-0.29cm] frame.north west) --
        ([yshift=-0.29cm, xshift=0.2\linewidth] frame.north west);
    }
  ]
  \hspace{0.20\linewidth}
  \textbf{\texttt{\Large{#1}}}
  \large
  \vspace{5pt}
  \newline
}
{ \end{tcolorbox} }

\newenvironment{tightemize}
{\vspace{-2pt}\begin{itemize}[leftmargin=*]\itemsep4pt \parskip0pt \parsep0pt}
{\end{itemize}}

\newcommand{\HRule}[2]{\textcolor{#1}{\rule{\linewidth}{#2}}}
\newcommand{\header}[1]{{\large\textbf{#1}}\newline}
\newcommand{\locationdate}[2]{\textit{#1} \hfill \textit{#2}}
\newcommand{\spacer}{\quad}

\begin{document}
\thispagestyle{empty}
\begin{center}
  { \huge \texttt{samuel weitzman}}\\[0.2cm]
\foreach \x in {C5, C4, C3, C2, C1} {%
  {\color{\x}\rule{.11\textwidth}{0.8pt}}%
}\\[5pt]
{\large
  {\textcolor{black!75}{\large\faPhone}}\ \texttt{774.218.5184}\spacer
  {\textcolor{black!75}{\large\faEnvelope}}\ \texttt{sameweitzman@gmail.com}\spacer\\[5pt]
  {\textcolor{black!75}{\large\faHome}}\ \href{http://almostsam.com}{\texttt{almostsam.com}}\spacer
  {\textcolor{black!75}{\large\faLinkedin}}\ \href{http://linkedin.com/in/seweitzman}{\texttt{linkedin.com/in/seweitzman}}\spacer
  {\textcolor{black!75}{\large\faGithub}}\ \href{http://github.com/sweitzma}{\texttt{github.com/sweitzma}}\\
}
\foreach \x in {C5, C4, C3, C2, C1} {%
  {\color{\x}\rule{.18\textwidth}{0.6pt}}%
}
\end{center}

\vspace{1pt}
\begin{cvsection}{work - \href{https://aquabyte.ai/}{aquabyte}}{C1}
  \header{ Senior Machine Learning Engineer }
  \locationdate{San Francisco, CA}{Feb. 2022 - Present}
  \begin{tightemize}
    \item Researched computer vision models for classification, keypoint detection, and
      3D geometry to weigh salmon from a stereo camera system.
    \item Deployed multiple models to both AWS cloud environments and edge devices.
    \item Models are active in production running inference on millions of images a day.
    % \item Led load testing and setting up networking in AWS to deploy models.
    % \item Occasional modeling work on more tabular data with non-deep learning techniques (RANCAC regression, xgboost, decision trees.)
    % \item Wrote a data quality model which significantly reduced cost and wasted bandwidth.
    % \item Created a model training template used across multiple teams to accelerate projects and encourage best practices.
    \item Comfortable working on all parts of the stack: modeling, data pipelines, infrastructure, backend, and frontend.

% key things to highlight
% - deep learning and computer vision
% - deploying to edge and cloud
% - also ran




  \end{tightemize}
\end{cvsection}
\begin{cvsection}{work - \href{https://meraki.cisco.com/}{cisco meraki}}{C1}
  \header{ Machine Learning Engineer }
  \locationdate{San Francisco, CA}{Sep. 2018 - Jan. 2022}
  \begin{tightemize}
    % \item Led, built, and deployed network usage forecasting feature
    % \item Developed multiple anomaly detection models to uncover networking problems
  \item Developed anomaly detection and timeseries forecasting models for networking issues and
    bandwidth use.
  \item Inventor on patent for internet outage detection
    (\href{https://patents.google.com/patent/US11398970B2}{US11398970B2}).
    \item Collaborate closely with PM to define product decisions and roadmap.
    % \item Enjoy being domain driven and using that to assist modeling or simplify solution


% patent: https://patents.google.com/patent/US11398970B2/en?inventor=samuel+weitzman


%     \item Developed multiple anoamly detection features on networking timeseries data (one in production right now)
%     \item Lead data analytics and project planning for multiple data driven features in the works
%     \item Collaborate with ML Infra team to define and encourage best practice ML workflows
%     \item Run a machine learning presentation meetup to foster collaboration and knowledge sharing across teams 

    % \item Developed a production anomaly detection system for web application perfromance issues
    % \item Improved user alert response by 18\%
    % \item Used Python, Pandas, Jax, NumPyro, Docker, AWS, Terraform, and MLFlow
    % \item Actively involved in model development and optimization, developer workflows and tools, and productionizing.
    % \item Mentor to full time hires and adviser for intern research projects
    % \item Worked on Meraki web app features and scalability, writing Ruby on Rails / Scala backend
      % and JavaScript + React frontend
    % \item Close collaboration with PM and UX to scope new ML features and product decisions
    % smart thresholds, load balancers, code profiling alerts, mlflow
  \end{tightemize}
\end{cvsection}
\vfill
\begin{cvsection}{education - \href{https://www.brown.edu/}{brown university}}{C2}
  \header{B.Sc. Mathematics - Computer Science}
  \locationdate{GPA: 4.00}{Sep. 2014 - May 2018}\\
  % Diverse undergrad education specializing in topics at the intersection of pure math, applied math, and computer
  % science.
  % Capstone in computer vision.\\
\end{cvsection}
\vfill
\begin{cvsection}{about me}{C3}
I want my work to solve critical climate and environmental problems.
I thrive with a challenging technical problem and owning a project from research to deployment. 
  Outside of tech my interests include sustainable agriculture, trail running, and cooking with friends.\\
\end{cvsection}
\vfill
\begin{minipage}[t]{.25\textwidth}
  \begin{cvsection}{languages}{C4}
    Python\skill{4}
    Ruby\skill{4}
    SQL\skill{3}
    Bash\skill{2}
    Javascript\skill{2}
    \TeX\skill{2}
  \end{cvsection}
  \vfill
\end{minipage}
\hspace{2cm}
\begin{minipage}[t]{.25\textwidth}
  \begin{cvsection}{libraries}{C4.5}
    pandas\skill{4}
    pytorch\skill{3}
    sklearn\skill{3}
    pytest\skill{3}
    OpenCV\skill{2}
    dask\skill{2}
  \end{cvsection}
  \vfill
\end{minipage}
\hspace{2cm}
\begin{minipage}[t]{.25\textwidth}
  \begin{cvsection}{tools}{C5}
    Docker\skill{4}
    Git\skill{4}
    AWS\skill{3}
    Databases\skill{2}
    MLFlow\skill{2}
    Terraform\skill{1}
  \end{cvsection}
  \vfill
\end{minipage}
\end{document}
